\documentclass[preprint,10pt]{elsarticle}
\usepackage{etoolbox}
\makeatletter
\patchcmd{\ps@pprintTitle}{\footnotesize\itshape
       Preprint submitted to \ifx\@journal\@empty Elsevier
       \else\@journal\fi\hfill\today}{06.2018}{}{}
\makeatother

\usepackage[margin=2.5cm]{geometry}

\newcommand{\fscale}[1]{#1\linewidth}
\newcommand{\figref}[1]{Fig.~\ref{#1}}

%-----------------------------------------------------------------------------
%  Colors
%-----------------------------------------------------------------------------
\usepackage{pagecolor}
\definecolor{linen}{RGB}{240,240,230}
\definecolor{fwhite}{RGB}{255,250,240}
\definecolor{oldlace}{RGB}{253,245,230}
\definecolor{awhite}{RGB}{250,235,215}
\definecolor{pwhip}{RGB}{255,239,213}
\definecolor{peach}{RGB}{255,218,185}
\definecolor{ivory}{RGB}{255,255,240}
\definecolor{seashell}{RGB}{	255,245,238}

\usepackage{sectsty}
\sectionfont{\Large \textsf}
\subsectionfont{\large \textsf}

\usepackage{amsmath}
\newcommand{\mr}[1]{\mathrm{#1}}  % Roman font for the math mode
\newcommand{\mi}[1]{\mathit{#1}}  % Italic font for the math mode
\newcommand{\mc}[1]{\mathcal{#1}} % Caligrafic (script) font for the math mode
\newcommand{\ms}[1]{\mathsf{#1}}  % Sans font for the math mode (vectors and matrices)
\newcommand{\mb}[1]{\mathbf{#1}}  % Bold font for the math mode (vectors and matrices)

\usepackage{mathtools}
\DeclarePairedDelimiter\ceil{\lceil}{\rceil}
\DeclarePairedDelimiter\floor{\lfloor}{\rfloor}

\usepackage{amsthm}
\theoremstyle{definition}
\newtheorem{definition}{Definition}[section]
\newtheorem{remark}{Rule}[section]

\newtheorem{property}{Property}

\newtheorem{theorem}{Theorem}

\usepackage{cleveref}
\crefformat{section}{\textbf{\S#2#1#3}} % see manual of cleveref, section 8.2.1
\crefformat{subsection}{\textbf{\S#2#1#3}}
\crefformat{subsubsection}{\textbf{\S#2#1#3}}

\renewcommand{\cref}[1]{\textbf{\S\ref{#1}}}

\usepackage{graphicx}
\usepackage{subfigure}
\graphicspath{{./fig/}}
%% The amssymb package provides various useful mathematical symbols
\usepackage{amssymb}

\usepackage[titletoc]{appendix}
\usepackage[T1]{fontenc}

\usepackage{xcolor, soul}
\sethlcolor{oldlace}

\usepackage{lineno}
\usepackage{url}
\bibliographystyle{unsrt}

\patchcmd{\abstract}{Abstract}{Summary}{}{}

\begin{document}

\begin{frontmatter}

%% Title, authors and addresses

\title{\textsf{Latitude: A Blockchain for the Transportation Industry\tnoteref{title1}}}
\tnotetext[title1]{Version 1.1}

\author{Arunesh Mishra, Latitude Labs\corref{auth1}}
\cortext[auth1]{Advisor/Consultant, Latitude Labs at https://latitude0x.com}
\address{Silicon Valley, California}

\begin{abstract}

Blockchains are rapidly becoming the new vehicles for decentralized applications for the data sharing economy. By their
    design, they are able to create privacy aware, secure, trusted, verifiable and user incentivized ways of sharing
    data. This ability has created a new wave of applications in various industries which are disrupting the status quo
    of the incumbents. Also, the Transportation industry is seeing a shift in terms of the amount of data that has
    become available due to mobile phones, dedicated sensors and crowdsourced availability of data. The industry has a
    large number of applications, users and regulations which can benefit from data sharing if it can be done correctly.

 We present the Latitude blockchain which is designed to be the best blockchain for transportation applications.
    Latitude is designed using principles derived from the core decentralized blockchain technologies available today
    combined with fundamental considerations for the unique aspects of Transportation data and applications. Latitude
    includes the world's first smart contract system specifically tailored for geo-spatial, mapping, location and sensor
    data and related applications. The system is powered by a geo-spatial decentralized datastore which can scale to
    planet level storage capabilities incentivized by a token economics built over the LAT (Latitude Token).

    The Latitude network is also provisioned with a library of algorithms, both open source and closed form, which allow
    Government, insurance companies and regulators to compute, certify and create cryptographic proofs which can help enforce policies
    or verify contracts.  Latitude has a decentralized Governance model using a Council system which is resistant to
    collusion and Byzantine behavior and is incentivized to maintain honest and proper operation of the network.

    Latitude can support four classes of applications. First, Ridesharing applications consist of the sharing economy
    applications for multi-modal ridesharing and transport. Using Latitude, it becomes possible to create better
    incentives for ride providers and users including incentives for data sharing and route optimizations. This also
    allows regulatory bodies to participate in the execution of these applications using smart contracts. Second,
    Telematics applications such as Usage-based Insurance (UBIs) and fleet management can benefit from real-time or
    historial data and corresponding intelligent algorithms, thus providing shared mutual benefits. The third class of
    application include mapping and location.  Using tangible user incentives while providing strong trust, privacy and
    security for user data, it is possible to create new or better data sharing applications (similar to Waze among
    others) on the Latitude platform. Finally, all the geo-spatial, location, mapping and real-time city level data can be used by the City and State
    governments for smart city applications to provide a better quality of life for their citizens.
\newline \newline \newline \end{abstract}

\begin{keyword}
	\textsf{location \sep cryptography\sep decentralization \sep blockchain \sep ethereum \sep database \sep sql
    \sep access control \sep p2p \sep tokens \sep incentives \sep mechanism design \sep byzantine
    faults \sep data sharing \sep latitude \sep longitude \sep driverscore \sep mapping \sep transportation \sep
    insurance}
\end{keyword}

\end{frontmatter}

\newpage
\tableofcontents
\newpage

%-----------------------------------------------------------------------------
%  INTRODUCTION
%-----------------------------------------------------------------------------
\section{Introduction}\label{sec:intro}

There is a fundamental shift happening in the transportation industry due to the availabilty of data and resulting
applications in ways that was not possible before. For instance, due to the proliferation of mobile phones and real-time
location data, it became possible to disrupt the traditional taxi-based transportation market using Ridesharing
applications. Similar trend in real-time traffic and incident data collected in a crowdsourced manner has led to apps
like Waze which have improved our daily commute life. Using data-sharing it also became possible to build apps that can
do multi-modal ride computations (bike, scooters, etc).

\noindent
{\textsf Benefits of Data sharing:}

%Big data has become the fuel for Decision making, product evolution, AI applications, including the use of Deep Learning
%and other techniques.

Users of these apps are providing a staggering amount of data. However, due to centralization, this data is subject to
policies, security and privacy practices that are not in the user's control. Morever, its not possible for regulators or
city governments to use this data for benefit of their citizens. The recent breaches in data usage, security and privacy
have shown that users want control over who is using their data and how. This includes how advertisers are using their
data, for example. Also, regulators want to make sure policies are implemented correctly.

Now, imagine if these hurdles can be taken care of using the right technology, would data sharing enable new and
disruptive applications ? The possibility of enabling data sharing using blockchain technology with cryptographically
protected sharing methods has the potential to change the sharing economy.

We present Latitude which is designed to be the best blockchain for the Transportation industry. Latitude allows the
development of decentralized apps using smart contracts powered by a geo-spatial datastore. This is built off strong
cryptographic proofs and primitives that provide security, privacy and anonymity guarantees.  For the Transportation
Industry, such data sharing can help solve some critical problems.  For example, cities can share the data they
collected about transport patterns with different stake holders such as transportation companies to reduce the traffic
jams during rush hour \cite{traffic_jam}. First responders and aid workers share data between them to better coordinate
disaster relief efforts \cite{bharosa_2010}.

- Centralized solution examples:
    - Telematics data exchanges.
    - Mapping and location data: privacy security issues. Lack of user control.
    - Centralization creates DATA SILOS. Explain and expand.
- What powers this is data, being able to share data, analytics, computation using AI, in a trustworthy privacy-aware
manner.

- The industry needs a solution that can allow new ways of sharing the data while providing strong unbreakable gurantees
of security privacy etc.

\noindent
{\textsf What are blockchains:}
Blockchain is a public register in which transactions between two users belonging to the same network are stored in a secure, verifiable
and permanent way. The data relating to the exchanges are saved inside cryptographic blocks, connected in a hierarchical
manner to each other. This creates an endless chain of data blocks -- hence the name blockchain -- that allows you to
trace and verify all the transactions you have ever made.

Bitcoin was the first such blockchain to introduce this concept, but Ethereum has made this a general platform for
executing arbitrary applications called dapps using smart contracts and a wide range of other tools. Since Ethereum,
there has been an explosion in blockchain technology to allow a wide range of distributed decentralized applications to
run over a network of untrusted arbitrary nodes over the planet, which almost mimic the structure and spread of the
Internet.

\noindent
{\textsf Related projects:}
Projects such as the Origin Protocol, Uchain among others are building platforms that empowers developers and businesses
to create their own decentralized marketplaces for sharing data in accordance with strict controls on the blockchain.
Such Blockchain based approaches make it quick and easy for organizations to develop and manage listings for assets and
services. Similar data sharing can yield benefits in the Internet of Things (IoT) landscape. Projects such as the IoTex
blockchain are building solutions to collect, store and share data from IOT sensors, protected with a smart contract
system.

The rest of this whitepaper is organized as follows. In the next Section, we discuss why Blockchains are the proper
technology to disrupt the Transportation industry. In Section \ref{sec:design}, we present the design of the Latitude
blockchain, including the core building blocks that make it work well for applications in the transportation industry.
In Section \ref{sec:apps}, we discuss details of four different classes of applications that can be built on Latitude
and how third-party companies can build decentralized applications to disrupt the incumbents.

- Overview of our solution:
  - Blockchains can provide these gurantees and much more. 
  - Examples of other industries where blockchain is posed to disrupt traditional data sharing:
    - IOT example.
    - Supply chain management.
    - Renting.
  - Section level overview.



We are in the midst of building a new Internet. Blockchain networks like Ethereum have popularized the idea of
unstoppable, owner less applications that are run on an open network of untrusted but incentivized nodes without the
oversight of a central authority. An increasing number of these decentralized applications (\DJ apps) are being created
everyday with evolving data storage needs. The first generation of these dapps stored their data entirely on a
blockchain itself while the current ones are storing it on decentralized file storage systems like IPFS with just hashes
of the data stored on the blockchain. Although this method works for dapps with simple storage needs like the need for
storing data as a blob, it is very limiting for dapps with more complex requirements such as the need to store data at a
more fine grained level and efficiently querying it. A popular option is then to use a cloud hosted database (ex:
Google’s Cloud SQL) but that turns a dapp into a non-dapp by introducing centrality into the system. \newline\newline

Rest of the paper is structured as follows: in section 2, we discuss some related work. Section 3 presents the overall design of the system, section 4 presents the network subsystem, section 5 presents the database subsystem and section 6 presents our approach to dealing with problems that arise in a network made up of untrusted nodes and discusses an incentive mechanism to make them behave as per system's needs. In section 7, we propose a message exchange protocol in the same vein as http but for relational data sharing. Section 8 concludes the paper.


The introduction of Bitcoin \cite{nakamoto2009bitcoin} has triggered a new wave of decentralization in computing. 
Bitcoin illustrated a novel set of benefits: decentralized control, where ``no one'' owns or controls the network; immutability,
where written data is tamper-resistant (``forever''); and the ability to create \& transfer assets on the network, without reliance on a central entity.

The initial excitement surrounding Bitcoin stemmed from its use as a token of value, for example as an alternative to government-issued currencies.
As people learned more about the underlying blockchain technology, they extended the scope of the technology itself (e.g. smart contracts), as well as applications (e.g. intellectual property).

With this increase in scope, single monolithic ``blockchain'' technologies are being re-framed and refactored into building blocks at four levels of the stack:
\begin{enumerate}
 \item Applications
 \item Decentralized computing platforms (``blockchain platforms'')
 \item Decentralized processing (``smart contracts'') and decentralized storage (file systems, databases), and decentralized communication
 \item Cryptographic primitives, consensus protocols, and other algorithms
\end{enumerate}


\section{Why use a Blockchain ?}\label{sec:blockchain}

In this Section, we go over the key reasons why a Blockchain-based platform is the right solution for Latitude. We
start with an overview of the Blockchain technologies and present the pivotal features that would work well for the
business use case of Latitude.

Blockchain has the potential to become the new decentralized application platform for the Internet. Blockchain is a
public register in which transactions between two users belonging to the same network are stored in a secure, verifiable
and permanent way. The data relating to the exchanges are saved inside cryptographic blocks, connected in a hierarchical
manner to each other. This creates an endless chain of data blocks -- hence the name blockchain -- that allows one to
trace and verify all the transactions they have ever made.

The introduction of Bitcoin \cite{nakamoto2009bitcoin} triggered a new wave of decentralization in computing.
Bitcoin illustrated a novel set of benefits: decentralized control, where ``no one'' owns or controls the network;
immutability, where written data is tamper-resistant (``forever''); and the ability to create \& transfer assets on the
network, without reliance on a central entity.

The initial excitement surrounding Bitcoin stemmed from its use as a token of value, for example as an alternative to
government-issued currencies.  As people learned more about the underlying blockchain technology, they extended the
scope of the technology itself (e.g. smart contracts), as well as applications (e.g. intellectual property).

Bitcoin was the first such blockchain to introduce the concept of full decentralization, but Ethereum has made this a
general platform for executing arbitrary applications called dapps using smart contracts and a wide range of other
tools. Since Ethereum, there has been an explosion in blockchain technology to allow a wide range of distributed
decentralized applications to run over a network of untrusted arbitrary nodes over the planet, which almost mimic the
structure and spread of the Internet.

%Blockchain is a
%public register in which transactions between two users belonging to the same network are stored in a secure, verifiable
%and permanent way. The data relating to the exchanges are saved inside cryptographic blocks, connected in a hierarchical
%manner to each other. This creates an endless chain of data blocks -- hence the name blockchain -- that allows you to
%trace and verify all the transactions you have ever made.

 The primary function of a blockchain is, therefore, to certify transactions between people. In the case of Bitcoin, the
 blockchain serves to verify the exchange of cryptocurrency between two users, but it is only one of the many possible
 uses of this technological structure. In other sectors, the blockchain can certify the exchange of shares and stocks,
 operate as if it were a notary and "validate" a contract or make the votes cast in online voting secure and impossible
 to alter.

Decentralization is one of the core concepts or features of a blockchain. It can mean different things in different
contexts but for our purposes it allows for two important things:
\begin{itemize}
    \item Decentralization of control or power: That is, no single entity such as a company or an institution has
        unrestricted control over all aspects of the data. In a centralized world, for example, Uber has direct control
        over all user data and how that data is being shared with third-parties, etc. Of course, there are privacy
        policies that are published, but as a user one has no choice but to place full trust in the policy or their
        implementations. There is no method of open verification or recourse in case of a breach. With a
        blockchain-based solution, all participants on the network have equal say. Using primitives such as consensus
        and smart contracts, it becomes possible to verify and enforce such policies.
    \item Decentralization of software: There is no single centralized place on the Internet which hosts the
        functionality or the data. The blockchain itself is stored in a decentralized manner on the nodes that form the
        network and thus, there is no single point of failure or trust associated with the system. This is a big
        advantage that blockchain systems have over centralized solutions.
\end{itemize}

As a result of this decentralization, the blockchains get some nice benefits as given below:
\begin{itemize}

    \item Fault tolerance - decentralized systems are less likely to fail accidentally because they rely on many separate
components that have uncorrelated failure models.
    \item Attack resistance - decentralized systems are more expensive to attack and destroy or manipulate because they lack
sensitive central points that can be attacked at much lower cost than the economic size of the surrounding system. This
can be important for transportation data as it does not remain under a single point of failure.
    \item Collusion resistance - it is much harder for participants in decentralized systems to collude to act in ways that
benefit them at the expense of other participants, whereas the leaderships of corporations and governments collude in
ways that benefit themselves but harm less well-coordinated citizens, customers, employees and the general public all
the time.
\end{itemize}

One of the key concepts of blockchains becoming an application platform is smart contracts. A standard contract, as a
legal document, binds two or more parties into an obligation to achieve certain deliverables or outcomes. A smart
contract is a piece of code that similarly binds multiple parties into outcomes that are verifiable, computable or
provable using code and strong cryptographic constructs. Ethereum was the first such platform to introduce the concept
of smart contracts which has since been adopted by most other blockchains that wish to host decentralized applications
(dapps).

Because smart contracts can be executed by arbitrary nodes on the blockchain, its possible for anyone to "verify" the
smart contract. This creates the concept of trust using consensus. Consensus is defined as the agreement among a
certain number (or fraction) of nodes on a particular result or outcome. With consensus, it becomes much harder for a
Byzantine or adversarial node \cite{lamport_byz} to manipulate the smart contract in ways that was not intended or
provisioned for. One of our goals, as discussed later in Section \ref{sec:design} is to build a smart contract framework
that is tailored for Transportation applications, so that the contracts are readily available, trustable and
enforceable. 

Smart contracts can allow new ways of data sharing that were not possible before. By creating strong programmatic
constructs combined with consensus protocols and cryptographic primitives it is possible to share data in a way that
privacy, security and restrictions on use can be enforced. There has been a lot of recent progress on how to use Smart
Contracts for data sharing \cite{liu_2018}. Thus, the technology today is ready for creating disruptive new ways of
sharing transportation data to create new applications such as smart cities, driver behavior, insurance, mapping etc.
This technology can be disruptive to incumbents who might be late for adoption.

Blockchain software is generally open-source. This increases the trust level to what is not possible in centralized data
silos.  Open source software removes the need for policies to exist in text, but can be verified in code. Also it allows
anyone in the community or the industry to contribute to functionality thereby moving the industry towards
standardization which is good for the ecosystem. This makes it possible for the Industry or Government to create
regulations or an industry standard and also implement a method of enforcing them on the network.

%- Concepts of smart contract, trust, consensus.
%
%- How data sharing can work in a blockchain manner. 
%- Overview of cryptographic proofs. How they work.
%- Discussion of Byzantine behavior.
%
%-Blockchain based Platform is decentralized.
%    - No single point of control or failure.
%    - Users own their data and reputation.
%    - Every functionality is openly verifiable.
%
%- System is open-source
% - Every entity knows how data, algorithms and logic is implemented.
% - Policies can be verified using Trusted Computing.
%
%- Anyone can participate including Governments, Individuals, Enterprises.
%- Governance happens using a council of participants.
%- Privacy and Security policies can be enforced through strong cryptographic primitives.
%
%
%- Smart contracts allow for precise enforcement of incentives, data sharing and privacy protection.
%- Industry or Government Standards are enforceable and verifiable.
%- Raw and derived data can outlive the companies or geographies.
%- Eg: DriverScore can carry over to a different country or geography.
%- Mapping data can be utilized outside of the GIS provider.
%
%


\section{The Latitude Blockchain}\label{sec:design}


Overview of design principles:

Datastore
 - Optimized for Geographic, Geo-spatial data.
 - Location, Mapping data and computation.
 - Ability to Store, index, query and build smart contracts optimized for such data.
 - Support for various spatial indexes.
 - Location heatmaps.
 - Indexing road and driving data.
 - Primitives for storing driving data for autonomous vehicles

Cryptographic primitives for:
 - Security and privacy of data.
 - Anonymity guarantees using cryptographic set operations.
 - Enforcement of privacy when sharing data.
 - Sharing of “computation” instead of data when possible.
 - For eg: Sharing of DriverScore using a vetted algorithm.
 - Sharing proximity to a landmark instead of lat/lng.
 - Ability to find bad actors.
 - Detect privacy, anonymity and security violations.


Smart contract system for Transportation applications.
 - Ability to convert “policies” such as GDPR into smart contract code.
 - Example, self destruct data after a time period.
 - Sandboxed trusted execution environment:
 - For algorithms:
   - DriverScore, Location heatmaps, Statistics.
   - Enforcing or verifying privacy and other govt policies/regulations.

Cryptographic proofs for applications:
 - Proof of Location. 
 - Proof of ride. 
 - Proof of mapping 
      (road/landmark exists or does not exist).
 - Proof of driver score 
 - Open, trusted, understood driver score computation algorithms.
 - Cryptographic proofs can be shared among entities, safely, securely.


Cryptoeconomics and Governance:
 - Crypto-incentives for honest operation.
 - Penalties for malicious intent.
 - Governance based on consensus and roles using a council.
 - Council members elected using voting, stake and established trust.
 - Some council members can have restricted access.
 - Eg: US govt can have voting rights on US data/users, etc.

User incentives:
 - Users have full control over their data and computation.
 - Users can issue or request proofs to carry over to other applications.
 - User’s have incentive to share data, participate in improving the common denominator.
 - Malicious intent, Byzantine behavior.
   - Can be detected using a combination of consensus and incentives.
   - Best interest of users to act honestly by design.

\subsection{Blockchain Architecture}



Performance:
- Transaction speed. Data throughput. Storage capabilities.


\section{Applications}
\label{sec:apps}

\subsection{Rideshare applications}

Overview of ridesharing applications. Big industry leaders. Statistics on rides, miles, revenue, ARPU.

Multi-modal ride data:
 - Bike, scooter ride apps.
Disruptive to Bikesharing:
 - Mobike, OFO, BlueGoGo, Youon, Mingbikes
 - Hellobike, YooBike, CCbike, Zagster, LimeBike
 - Citi Bike, Capital Bikeshare, Divvy, Hubway, Docomo Bike
Share, Relay Bikes
 - Public transit data
Use SDK/app to share data. Similar incentive model.
Monetization:
 - Single multi-modal ride API (SherpaShare or 3rd party).

 Ridesharing platform:

 Platform:
Latitude Ridesharing SDK.
Ridesharing sidechain on the Latitude blockchain.
Riders:
Contributing ridesharing data:
Uber, Lyft, Gett, Juno, Curb, Sitbaq
Drivers:
 - Use SherpaShare app or 3rd-party app with SDK.
 - Proof of ride:
Using driver-side app and/or client app.
Other players:
 - City, Law enforcement, analytics use cases.

\subsection{Telematics applications}


\subsection{Mapping and Location}

\subsection{Smart city and Govt Applications}


%\input{mech}

\section{Conclusion}
\label{sec:conc}

Blockchains have the potential to disrupt incumbent applications on the data sharing economy. In addition to this, the
Transportation industry is going through a radical shift due to the new types of data and applications that have become
available, for instance, due to mobile phones, crowdsourced data, cheap hardware sensor networks and sophisticated
analysis and planning algorithms that can utlize this data.

By bring the blockchain technology and designing it from the grounds up for Transportation applications we have created
the worlds first blockchain specifically tailored for Transportation applications. Our mission is for Latitude to become
the de-facto platform for all transportation applications by building the right constructs for trusted, privacy aware,
secure and verifiable data and computation sharing/enforcement.

Our mission is to build Latitude to support applications across the planet and across different modes of transport. We
are excited by how Latitude stands to disrupt existing applications such as Ridesharing, mapping, location sharing and
analytics and the driver-behavior industry (UBIs).

The full-whitepaper shall contain a deeper dive into the mechanics of the Latitude blockchain including details on how
the smart contract system, the cryptographic proofs and the datastore would function.


% One section per vertical.

% Core blockchain aspects. Smart Contract system. DataStore etc.

% Governance. Consensus, cryptoeconomics.

%-----------------------------------------------------------------------------
%  OVERALL DESIGN SECTION
%-----------------------------------------------------------------------------
%\section{The Latitude Blockchain}\label{sec:design}


Overview of design principles:

Datastore
 - Optimized for Geographic, Geo-spatial data.
 - Location, Mapping data and computation.
 - Ability to Store, index, query and build smart contracts optimized for such data.
 - Support for various spatial indexes.
 - Location heatmaps.
 - Indexing road and driving data.
 - Primitives for storing driving data for autonomous vehicles

Cryptographic primitives for:
 - Security and privacy of data.
 - Anonymity guarantees using cryptographic set operations.
 - Enforcement of privacy when sharing data.
 - Sharing of “computation” instead of data when possible.
 - For eg: Sharing of DriverScore using a vetted algorithm.
 - Sharing proximity to a landmark instead of lat/lng.
 - Ability to find bad actors.
 - Detect privacy, anonymity and security violations.


Smart contract system for Transportation applications.
 - Ability to convert “policies” such as GDPR into smart contract code.
 - Example, self destruct data after a time period.
 - Sandboxed trusted execution environment:
 - For algorithms:
   - DriverScore, Location heatmaps, Statistics.
   - Enforcing or verifying privacy and other govt policies/regulations.

Cryptographic proofs for applications:
 - Proof of Location. 
 - Proof of ride. 
 - Proof of mapping 
      (road/landmark exists or does not exist).
 - Proof of driver score 
 - Open, trusted, understood driver score computation algorithms.
 - Cryptographic proofs can be shared among entities, safely, securely.


Cryptoeconomics and Governance:
 - Crypto-incentives for honest operation.
 - Penalties for malicious intent.
 - Governance based on consensus and roles using a council.
 - Council members elected using voting, stake and established trust.
 - Some council members can have restricted access.
 - Eg: US govt can have voting rights on US data/users, etc.

User incentives:
 - Users have full control over their data and computation.
 - Users can issue or request proofs to carry over to other applications.
 - User’s have incentive to share data, participate in improving the common denominator.
 - Malicious intent, Byzantine behavior.
   - Can be detected using a combination of consensus and incentives.
   - Best interest of users to act honestly by design.

\subsection{Blockchain Architecture}



Performance:
- Transaction speed. Data throughput. Storage capabilities.


%----------------------------------------------------------------------------- BIBLIOGRAPHY
%-----------------------------------------------------------------------------
\section{Author Bio}

\noindent {\bf Arunesh Mishra:} Dr. Mishra has 10 years of research and development experience in the Industry in
Location and Mapping. At Google, he built the Location platform used widely on Android and Chrome and Google Maps
applications. He was the core designer for the crowdsourced algorithms that self-learn and power the location/mapping
for Google's mobile platform. The platform serves more than 2 Billion users daily. He holds a PhD, 30 issued patents and
22 pending patents today many of which are based on the physical layer mechanics of computing location and mapping. He
was won Best Paper awards at top ACM conferences and has over 30+ conference publications. He has widely contributed to
open source software used on standard Linux distributions today. He has also built the largest mobile peer-peer network
infrastructure for iOS/Android via Google Play Games offering.

\section{References}
\bibliography{latitude.bib}

\end{document}
