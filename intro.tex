\section{Introduction}\label{sec:intro}

There is a fundamental shift happening in the transportation industry due to the availabilty of data and resulting
applications in ways that was not possible before. For instance, due to the proliferation of mobile phones and real-time
location data, it became possible to disrupt the traditional taxi-based transportation market using Ridesharing
applications. Similar trend in real-time traffic and incident data collected in a crowdsourced manner has led to apps
like Waze which have improved our daily commute life. Using data-sharing it also became possible to build apps that can
do multi-modal ride computations (bike, scooters, etc). Smart cities are able to use transportation data for
urban planning, zoning and development use cases. Insurance companies are able to provide per-mile insurance using
driver behavior determined by data from sensors.

\noindent
{\bf \textsf Benefits of Data sharing:}

%Big data has become the fuel for Decision making, product evolution, AI applications, including the use of Deep Learning
%and other techniques.

Users of these applications are providing a staggering amount of data. However, due to centralization this data is
subject to policies, security and privacy practices that are not in the user's control. Morever, its not possible for
regulators or city governments to use this data for benefit of their citizens. Centralization creates data silos which
dramatically reduce the utility of the technology. The recent breaches in data usage, security and privacy have shown
that users want control over who is using their data and how. This includes how advertisers are using their data, for
example. Also, regulators want to make sure policies are implemented correctly but have no programmatic way of doing so.

Now, imagine if these hurdles can be taken care of using the right technology, would data sharing enable new and
disruptive applications? Lately, Blockchain technology has emerged as a way for creating decentralized data-based
applications in a secure, privacy-aware, verifiable and trusted manner. The possibility of enabling data sharing using
blockchain technology with cryptographically protected sharing methods has the potential to change the sharing economy.

We present Latitude, which is designed to be the best blockchain for the Transportation industry. Latitude allows the
development of decentralized apps using smart contracts powered by a geo-spatial datastore. This is built off strong
cryptographic proofs and primitives that provide security, privacy and anonymity guarantees.  For the Transportation
Industry, such data sharing can help solve some critical problems.  For example, cities can share the data they
collected about transport patterns with different stake holders such as transportation companies to reduce the traffic
jams during rush hour \cite{traffic_jam}. First responders and aid workers can share data between them to better coordinate
disaster relief efforts \cite{bharosa_2010}.

For the common user, Latitude can help build decentralized apps that can provide incentives to both drivers and users
for sharing their data. This creates new incentive structures for apps such as ride sharing or per-mile insurance that
were not possible before. While Latitude allows safe and incentivized sharing of data, it is also a platform for
building decentralized applications for the Transportation Industry. For example, a city can build applications on
Latitude to understand traffic patterns or regulate driver licenses. Centralized Telematics exchanges such a Verisk or
Octo can be fully decentralized and open to everyone. Their operations can be replaced by intelligent and
cryptographically strong smart contracts.

Latitude will provide a state-of-the-art environment for developers including the availability of a production
geo-spatial datastore, a library of smart contracts and a community to foster new application development. Latitude
shall include mobile SDKs to help build mobile apps that can directly tie into the platform. We believe that Latitude
has the potential to disrupt existing centralized applications in the Transportation Industry and create new ones that
were simply not possible before.

%
%- Centralized solution examples:
%    - Telematics data exchanges.
%    - Mapping and location data: privacy security issues. Lack of user control.
%    - Centralization creates DATA SILOS. Explain and expand.
%- What powers this is data, being able to share data, analytics, computation using AI, in a trustworthy privacy-aware
%manner.
%
%- The industry needs a solution that can allow new ways of sharing the data while providing strong unbreakable gurantees
%of security privacy etc.

%\subsection{What are blockchains}
%Blockchain is a public register in which transactions between two users belonging to the same network are stored in a secure, verifiable
%and permanent way. The data relating to the exchanges are saved inside cryptographic blocks, connected in a hierarchical
%manner to each other. This creates an endless chain of data blocks -- hence the name blockchain -- that allows you to
%trace and verify all the transactions you have ever made.
%
%The introduction of Bitcoin \cite{nakamoto2009bitcoin} triggered a new wave of decentralization in computing.
%Bitcoin illustrated a novel set of benefits: decentralized control, where ``no one'' owns or controls the network;
%immutability, where written data is tamper-resistant (``forever''); and the ability to create \& transfer assets on the
%network, without reliance on a central entity.
%
%The initial excitement surrounding Bitcoin stemmed from its use as a token of value, for example as an alternative to
%government-issued currencies.  As people learned more about the underlying blockchain technology, they extended the
%scope of the technology itself (e.g. smart contracts), as well as applications (e.g. intellectual property).
%
%Bitcoin was the first such blockchain to introduce concept of full decentralization, but Ethereum has made this a
%general platform for executing arbitrary applications called dapps using smart contracts and a wide range of other
%tools. Since Ethereum, there has been an explosion in blockchain technology to allow a wide range of distributed
%decentralized applications to run over a network of untrusted arbitrary nodes over the planet, which almost mimic the
%structure and spread of the Internet.

\noindent
{\bf \textsf Related projects:}

There are a host of projects that are leveraging the blockchain technology to disrupt the sharing economy such as
renting homes or cars (AirBnb on the blockchain), etc. Projects such as the Origin Protocol and Uchain among others are
building platforms that empower developers and businesses to create their own decentralized marketplaces for sharing
data in accordance with strict controls on the blockchain.  Such Blockchain based approaches make it quick and easy for
organizations to develop and manage listings for assets and services. Similar data sharing can yield benefits in the
Internet of Things (IoT) landscape. Projects such as the IoTex blockchain are building solutions to collect, store and
share data from IOT sensors, protected with a smart contract system.  Platin \cite{platin} is a project that focuses on
proof of location on the blockchain and can be used for applications involving location sharing in a trusted and secure
manner. Thus in many industry verticals it is now becoming apparent that a well-designed blockchain-based platform can
be disruptive to centralized incumbent players. Such a platform can also foster the creatiion new applications that were
simply not possible before. This is a win-win scenario for users, governments and regulators alike.

Till date, there has been a void for a decentralized planet-wide system for all transportation applications. Latitude is
designed to fill that void by providing core blockchain constructs tailored for transportation applications including
specific SDKs and software components for each specific industry vertical (such as Ridesharing, Usage-based insurance,
etc).

The rest of this whitepaper is organized as follows. In the next Section, we discuss why Blockchains are the right 
technology to disrupt the Transportation industry. In Section \ref{sec:design}, we present the design of the Latitude
blockchain, including the core building blocks that make it work well for applications in the transportation industry.
In Section \ref{sec:apps}, we discuss details of four different classes of applications that can be built on Latitude
and how third-party companies can build decentralized applications to disrupt the incumbents.


%We are in the midst of building a new Internet. Blockchain networks like Ethereum have popularized the idea of
%unstoppable, owner less applications that are run on an open network of untrusted but incentivized nodes without the
%oversight of a central authority. An increasing number of these decentralized applications (\DJ apps) are being created
%everyday with evolving data storage needs. The first generation of these dapps stored their data entirely on a
%blockchain itself while the current ones are storing it on decentralized file storage systems like IPFS with just hashes
%of the data stored on the blockchain. Although this method works for dapps with simple storage needs like the need for
%storing data as a blob, it is very limiting for dapps with more complex requirements such as the need to store data at a
%more fine grained level and efficiently querying it. A popular option is then to use a cloud hosted database (ex:
%Google’s Cloud SQL) but that turns a dapp into a non-dapp by introducing centrality into the system. \newline\newline
%

%Rest of the paper is structured as follows: in section 2, we discuss some related work. Section 3 presents the overall
%design of the system, section 4 presents the network subsystem, section 5 presents the database subsystem and section 6
%presents our approach to dealing with problems that arise in a network made up of untrusted nodes and discusses an
%incentive mechanism to make them behave as per system's needs. In section 7, we propose a message exchange protocol in
%the same vein as http but for relational data sharing. Section 8 concludes the paper.
