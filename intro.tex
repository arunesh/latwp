\section{Introduction}\label{sec:intro}

- There is a fundamental shift happening in the transportation industry. Automation, data, AI based analytics.

{\em Why Data sharing ?}

Big data has become the fuel for Artificial Intelligence applications, including the use of Deep Learning and other
techniques. But this data typically lives in silos since data sharing is rife with problems. But imagine if the problems
can be taken care of using the right technology, would data sharing enable new and disruptive applications ? For the
Transportation Industry, data sharing can help solve some critical problems.  For example, cities can share the data
they collected about transport patterns with different stake holders such as transportation companies to reduce the
traffic jams during rush hour \cite{traffic_jam}. First responders and aid workers share data between them to better
coordinate disaster relief efforts \cite{bharosa_2010}.

- Centralized solution examples:
    - Telematics data exchanges.
    - Mapping and location data: privacy security issues. Lack of user control.
    - Centralization creates DATA SILOS. Explain and expand.
- What powers this is data, being able to share data, analytics, computation using AI, in a trustworthy privacy-aware
manner.

- The industry needs a solution that can allow new ways of sharing the data while providing strong unbreakable gurantees
of security privacy etc.

{\em What are blockchains ?}
Blockchain is a public register in which transactions between two users belonging to the same network are stored in a secure, verifiable
and permanent way. The data relating to the exchanges are saved inside cryptographic blocks, connected in a hierarchical
manner to each other. This creates an endless chain of data blocks -- hence the name blockchain -- that allows you to
trace and verify all the transactions you have ever made.

Bitcoin was the first such blockchain to introduce this concept, but Ethereum has made this a general platform for
executing arbitrary applications called dapps using smart contracts and a wide range of other tools. Since Ethereum,
there has been an explosion in blockchain technology to allow a wide range of distributed decentralized applications to
run over a network of untrusted arbitrary nodes over the planet, which almost mimic the structure and spread of the
Internet.

- Overview of our solution:
  - Blockchains can provide these gurantees and much more. 
  - Examples of other industries where blockchain is posed to disrupt traditional data sharing:
    - IOT example.
    - Supply chain management.
    - Renting.
  - Section level overview.



We are in the midst of building a new Internet. Blockchain networks like Ethereum have popularized the idea of
unstoppable, owner less applications that are run on an open network of untrusted but incentivized nodes without the
oversight of a central authority. An increasing number of these decentralized applications (\DJ apps) are being created
everyday with evolving data storage needs. The first generation of these dapps stored their data entirely on a
blockchain itself while the current ones are storing it on decentralized file storage systems like IPFS with just hashes
of the data stored on the blockchain. Although this method works for dapps with simple storage needs like the need for
storing data as a blob, it is very limiting for dapps with more complex requirements such as the need to store data at a
more fine grained level and efficiently querying it. A popular option is then to use a cloud hosted database (ex:
Google’s Cloud SQL) but that turns a dapp into a non-dapp by introducing centrality into the system. \newline\newline

Rest of the paper is structured as follows: in section 2, we discuss some related work. Section 3 presents the overall design of the system, section 4 presents the network subsystem, section 5 presents the database subsystem and section 6 presents our approach to dealing with problems that arise in a network made up of untrusted nodes and discusses an incentive mechanism to make them behave as per system's needs. In section 7, we propose a message exchange protocol in the same vein as http but for relational data sharing. Section 8 concludes the paper.


The introduction of Bitcoin \cite{nakamoto2009bitcoin} has triggered a new wave of decentralization in computing. 
Bitcoin illustrated a novel set of benefits: decentralized control, where ``no one'' owns or controls the network; immutability,
where written data is tamper-resistant (``forever''); and the ability to create \& transfer assets on the network, without reliance on a central entity.

The initial excitement surrounding Bitcoin stemmed from its use as a token of value, for example as an alternative to government-issued currencies.
As people learned more about the underlying blockchain technology, they extended the scope of the technology itself (e.g. smart contracts), as well as applications (e.g. intellectual property).

With this increase in scope, single monolithic ``blockchain'' technologies are being re-framed and refactored into building blocks at four levels of the stack:
\begin{enumerate}
 \item Applications
 \item Decentralized computing platforms (``blockchain platforms'')
 \item Decentralized processing (``smart contracts'') and decentralized storage (file systems, databases), and decentralized communication
 \item Cryptographic primitives, consensus protocols, and other algorithms
\end{enumerate}


