
\noindent
\section{Applications}
\label{sec:apps}

In this section, we discuss the four different classes of applications that can be built on the Latitude platform. Each
of these applications is supported by a suite of software modules, smart contract addons and SDKs (mobile and desktop)
that utilize the core functionality offered by Latitud and provision it in unique ways to suit the specific class of
applications. The Latitude platform is extensible in the sense that its not limited to these four classes of
applications. For example, its possible to add additional applications such as for the shipping or the airline industry
as verticals on the platform. Next we discuss each of these applications and how they can be supported on the platform.

\noindent
\subsection{Rideshare applications}

Ridesharing applications are the pinnacle of the applications in the data sharing economy. They empower the user to
choose from a multi-modal ways of finding or providing rides. Latitude can provide the decentralized infrastructure to
store, share and build decentralized applications for their corresponding counterparts.

The ridesharing market alone is large enough to power the Latitude blockchain as its primary usecase. The ridesharing
market is around 17 Billion dollars with around 60 Million users. This does not include the multi-modal ride sharing
segment which is growing rapidly. This segment includes bike, scooter and such modes of transport. The ARPU is around 
293 dollars which is high enough to create user incentivized models for data sharing. The data generated by these users
currently sits in silos in the respective ride sharing apps, which can be unlocked and put to good use through such
mechanisms. 

Latitude shall provide a mobile SDK and a blockchain API specifically to suit ridesharing apps for sharing data. This
would allow the creation of decentralized ride sharing apps that can provide users with incentives to share data,
subsidize rides and provide a better deal for ride providers. They can also allow regulators to enforce the use of
roads, lanes, parking and other structures in accordance with city or neighborhood ordinances. Latitude's SDK shall also
provide a multi-modal ride API that such apps can use to request rides on the platform.

As an preferred third-party integration, Latitude shall integrate with SherpaShare's platform for drivers. This would
allow greater sharing of the geo-spatial and driving data collected through SherpaShare's driver app. This will be one
of the early launches for the beta version of the Latitude platform.

%Overview of ridesharing applications. Big industry leaders. Statistics on rides, miles, revenue, ARPU.
%
%Multi-modal ride data:
% - Bike, scooter ride apps.
%Disruptive to Bikesharing:
% - Mobike, OFO, BlueGoGo, Youon, Mingbikes
% - Hellobike, YooBike, CCbike, Zagster, LimeBike
% - Citi Bike, Capital Bikeshare, Divvy, Hubway, Docomo Bike
%Share, Relay Bikes
% - Public transit data
%Use SDK/app to share data. Similar incentive model.
%Monetization:
% - Single multi-modal ride API (SherpaShare or 3rd party).
%
% Ridesharing platform:
%
% Platform:
%Latitude Ridesharing SDK.
%Ridesharing sidechain on the Latitude blockchain.
%Riders:
%Contributing ridesharing data:
%Uber, Lyft, Gett, Juno, Curb, Sitbaq
%Drivers:
% - Use SherpaShare app or 3rd-party app with SDK.
% - Proof of ride:
%Using driver-side app and/or client app.
%Other players:
% - City, Law enforcement, analytics use cases.

\noindent
\subsection{Telematics applications}

Data is the most important asset for Telematics applications. Thus, sharing of data, computations and algorithms can be
beneficial to companies, users and the ecosystem in general. In fact, by using the Latitude blockchain it becomes
possible for regulators to enforce laws better while maintaining data privacy. This can result in lower crime rates and
incidents.

The important of data sharing here can be seen from the existence of {\em Telematics data exchanges} that are
centralized cloud-based exchanges which act as data brokers among users and insurance companies. As an example, consider
the Verisk Data Exchante (verisk.com) which allows the exchange of driver data to insurance companies. They share data
for all kinds of connected vehicles for underwriting, rating, and claims handling through Usage-based Insurance (UBI)
programs.  The exchange stores and processes telematics data of all types, volumes, and velocity from connected cars,
aftermarket hardware, or mobile solutions. A second example is the Octo Telematics company which also operates an
exchange that is used by over 100 insurance companies, has about 186 billion miles of driving data and gets 11 billion
new data points from 5.4 million connected cars and sensors every day.

The problem with such centralized solutions is that there is a central entity that extracts the fees and has direct
control over the data. Such centralization can result in security and privacy loss and lack of trust from all parties
involved. Latitude is directly poised to disrupt this market by providing a blockchain based platform for exchange of
data, computation, algorithms and resulted information. Using Latitude smart contracts it becomes possible to enforce
proper security, privacy and sharing of data in accordance with agreed upon incentives. This implies cleaner and better
sharing methods which can be a win-win for everyone in the industry including the regulators.

Latitude shall provide a Telematics SDK and blockchain library module to enable such applications. This would include a
suite of pre-existing smart contracts, datastore provisions and other software modules necessary to build decentralized
data exchanges which can eliminate the expensive data brokers. The SDK shall also include APIs for Usage-based Insurance
companies to act as data providers given appropriate involvement, permission and incentives for the users. Users can
also be incentivized to provide data directly to the platform by getting token rewards. Such a model also allows for
open experimentation with driver behavior algorithms such as DriverScore. It also becomes possible for drivers and users
to "carry" their score or reputation from one platform to another, thus preventing lock-in scenarios.

%---Telematics data sharing:
%Data sharing can result in better algorithms, better understanding among drivers.
%Telematics data can save lives:
%Allows government regulations to be enforced. Data privacy can be enforced.
%Driver’s own their driving history (score) and can choose to carry it.
%
%
%---UBI methods can be improved with data sharing and incentives.
%Example:
%Octo telematics
%20 million miles, 186 billion miles of driving data
%100 insurance companies (data is not shared).
%11 billion new data points daily from 5.4 million connected cars and sensors.
%Disruptive opportunity for data sharing.
%
%
%
%---Telematics data exchanges exist today !
%But centralized, unclear policies, enforcement.
% Lack of control.
% Example: Verisk Data Exchange (verisk.com)
% The first-of-its-kind  exchange draws driving data from all kinds of connected vehicles for underwriting, rating, and
% claims handling through UBI programs. 
% The exchange is scalable to normalize, process, and store telematics data of all types, volumes, and velocity from
% connected cars, aftermarket hardware, or mobile solutions.
% Finally, one exchange solves the many-to-many challenge by connecting automakers and telematics service providers
% (TSPs) to multiple insurers
%
%
% ---Insurance Vertical
% Latitude Driver SDK:
% Allows trusted capture of driving data.
% Platform for running DriverScore / Behavior algorithms on phone or in the cloud (or a combination). 
% Eg: Geico:
% Can run 3rd-party algorithms such as Geico’s driver score directly on the phone.
% Includes API for interacting with Geico’s app.
% Users mine currency by:
% Contributing driver data, driver scores.
% Users own the data, have full control.
%
% Insurance providers (Geico, Progressive, Farmers, etc)
% Use platform for computing metrics.
% Contribute data/and or crypto-currency incentivized.
% Always user permissioned.
% Other players:
% City, Law enforcement, analytics use cases.
% Car history use cases (better version of Carfax).


\noindent
\subsection{Mapping and Location}

Mapping and Location-based services and analytics is another big segment that Latitude can address through
decentralization. Mapping and real-time location/speed data is a huge market that apps like Waze are able to access.
However, the user incentives in Waze are limited to the Waze platform and cannot be carried over. Finally, the data that
users contribute also stays isolated to the Waze platform and cannot be used by the City or the National Transportation
Safety Board (NTSB) for altruistic purposes. To get a sense of the market size, today Waze has about 70 million users
with 500K volunteers who provide real-time data. A Waze user on average spends about 480 minutes per month on the
platform.

The Latitude blockchain can change all of data isolation and platform lock-in by providing strong incentives to the user
by rewarding them with crytocurrencies. The smart contract system can also be used to enforce proper sharing of data
with the right participants. The built-in primitives in Latitude for Byzantine behavior can be used to ensure honest
operation at all times.

Location-based services can be similarly decentralized on Latitude. Using the mobile Latitude SDK, it becomes possible
for users to directly contribute mapping and location data to the platform. This allows Latitude to construct the
various cryptographic proofs as discussed in Section \ref{sec:crypto}. Proof of location technology alone has the
potential to disrupt the "Check-ins" industry. The Facebook platform alone gets about 50 million check-ins per year. The
other major players here include Foursquare, Yelp and Google maps. The sharing of check-ins data on the Latitude
platform can enable new applications not previously possible due to the platform lock-ins.

%Waze example:
% - 70 Mil users, 500k volunteers share data. user spends 480 min/month.
%
%-- Upcoming mapping applications include:
% High-resolution data for self-driving technology.
% Road data, needs consensus and trust.
% Proof of location:
% Location enabled maps. 
% Foursquare checkins. Facebook checkins.
% Location-based access control.
% 3 BILLION location requests daily on Android
% 250M location-enabled service users.
%
%--Latitude Mapping + Location SDK
%Mobile SDK for trusted mapping + location.
%SherpaShare app + 3rd party apps.
%Users mine by:
%Contributing road, traffic, street, mapping data.
%Contributing user location
%Proof of location concept can replace “Check-Ins”
%Facebook has 50 million check-ins per year.
%City, law enforcement, analytics use cases.
%User has full privacy control.
%
%-- Applications:
%Waze like apps:
%By Latitude Mapping/Location Sidechain.
%Waze, OsmAnd, Maps.Me, HEREWeGo, 2GIS
%Yandex.Maps, Locus Maps.
% Navmil, NavIt, MapQuest.
% Crowdsourced maps.
% Authenticated location applications:
% Access control or privileges by virtue of location.
%
\noindent
\subsection{Smart city and Govt Applications}

Smart cities refer to a large set of applications that can improve the life of a citizen by providing better amenities.
Examples include real-time data for parking, data-driven urban planning, city-wide zoning research,
environmental location sensors, disaster management, transport/transit data (real-time and historical), smart transport
(shuttle/bikes), traffic light managment to name a few.

Latitude blockchain can host such smart city applications and collect data from their citizens. Users are incentivized
to provide data using token economics and the regulators or city officials can "purchase" data or results using smart
contracts. It might be possible to share aggregated data with careful focus on privacy/anonymity to State level or
Federal government for various regulatory reasons. 

%One of the final set of applications.
%Smart city applications:
% - Parking, data-driven urban planning, waste management, water software and analytics, environmental location sensors,
%   disaster management, transport/transit data (real-time and historical), smart transport (shuttle/bike),
%   connectivity citywide, grid/energy, traffic light management, security and survelliance.
%
%  Users mine by contributing data.
%  Govt requests access using crypto.
%  Use cases:
%  Census, traffic, analytics, city-wide zoning research.
%  Real-time understanding of citizen data.
%  Compliance:
%  Insurance, Ridesharing companies regulatory compliance enforced using smart contracts on the Latitude blockchain.
