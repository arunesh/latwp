\section{Why Blockchains}\label{sec:blockchain}

Blockchain has the potential to become the new decentralized application platform over the Internet. Blockchain is a
public register in which transactions between two users belonging to the same network are stored in a secure, verifiable
and permanent way. The data relating to the exchanges are saved inside cryptographic blocks, connected in a hierarchical
manner to each other. This creates an endless chain of data blocks -- hence the name blockchain -- that allows you to
trace and verify all the transactions you have ever made.


 The primary function of a blockchain is, therefore, to certify transactions between people. In the case of Bitcoin, the
 blockchain serves to verify the exchange of cryptocurrency between two users, but it is only one of the many possible
 uses of this technological structure. In other sectors, the blockchain can certify the exchange of shares and stocks,
 operate as if it were a notary and "validate" a contract or make the votes cast in online voting secure and impossible
 to alter.
- Introduce the concept of decentralization, bitcoin, ethereum and blockchains in general.

Decentralization is one of the core concepts or features of a blockchain. It can mean different things in different
contexts but for our purposes it allows for (1) decentralization of control or power, that, (2) decentralization of
software.

{\em Why is decentralization useful ?}
\begin{itemize}

    \item Fault tolerance— decentralized systems are less likely to fail accidentally because they rely on many separate
components that are not likely.
    \item Attack resistance— decentralized systems are more expensive to attack and destroy or manipulate because they lack
sensitive central points that can be attacked at much lower cost than the economic size of the surrounding system.
    \item Collusion resistance — it is much harder for participants in decentralized systems to collude to act in ways that
benefit them at the expense of other participants, whereas the leaderships of corporations and governments collude in
ways that benefit themselves but harm less well-coordinated citizens, customers, employees and the general public all
the time.

\end{itemize}
- Concepts of smart contract, trust, consensus.

- How data sharing can work in a blockchain manner. 
- Overview of cryptographic proofs. How they work.
- Discussion of Byzantine behavior.

-Blockchain based Platform is decentralized.
    - No single point of control or failure.
    - Users own their data and reputation.
    - Every functionality is openly verifiable.

- System is open-source
 - Every entity knows how data, algorithms and logic is implemented.
 - Policies can be verified using Trusted Computing.

- Anyone can participate including Governments, Individuals, Enterprises.
- Governance happens using a council of participants.
- Privacy and Security policies can be enforced through strong cryptographic primitives.


- Smart contracts allow for precise enforcement of incentives, data sharing and privacy protection.
- Industry or Government Standards are enforceable and verifiable.
- Raw and derived data can outlive the companies or geographies.
- Eg: DriverScore can carry over to a different country or geography.
- Mapping data can be utilized outside of the GIS provider.


