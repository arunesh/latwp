\section{Background on Blockchain}\label{sec:blockchain}

Blockchain has the potential to become the new decentralized application platform over the Internet. Blockchain is a public register in which transactions between two users belonging to the same network are stored in a secure, verifiable
and permanent way. The data relating to the exchanges are saved inside cryptographic blocks, connected in a hierarchical
manner to each other. This creates an endless chain of data blocks -- hence the name blockchain -- that allows you to
trace and verify all the transactions you have ever made.

The introduction of Bitcoin \cite{nakamoto2009bitcoin} triggered a new wave of decentralization in computing.
Bitcoin illustrated a novel set of benefits: decentralized control, where ``no one'' owns or controls the network;
immutability, where written data is tamper-resistant (``forever''); and the ability to create \& transfer assets on the
network, without reliance on a central entity.

The initial excitement surrounding Bitcoin stemmed from its use as a token of value, for example as an alternative to
government-issued currencies.  As people learned more about the underlying blockchain technology, they extended the
scope of the technology itself (e.g. smart contracts), as well as applications (e.g. intellectual property).

Bitcoin was the first such blockchain to introduce concept of full decentralization, but Ethereum has made this a
general platform for executing arbitrary applications called dapps using smart contracts and a wide range of other
tools. Since Ethereum, there has been an explosion in blockchain technology to allow a wide range of distributed
decentralized applications to run over a network of untrusted arbitrary nodes over the planet, which almost mimic the
structure and spread of the Internet.

%Blockchain is a
%public register in which transactions between two users belonging to the same network are stored in a secure, verifiable
%and permanent way. The data relating to the exchanges are saved inside cryptographic blocks, connected in a hierarchical
%manner to each other. This creates an endless chain of data blocks -- hence the name blockchain -- that allows you to
%trace and verify all the transactions you have ever made.

 The primary function of a blockchain is, therefore, to certify transactions between people. In the case of Bitcoin, the
 blockchain serves to verify the exchange of cryptocurrency between two users, but it is only one of the many possible
 uses of this technological structure. In other sectors, the blockchain can certify the exchange of shares and stocks,
 operate as if it were a notary and "validate" a contract or make the votes cast in online voting secure and impossible
 to alter.
- Introduce the concept of decentralization, bitcoin, ethereum and blockchains in general.

Decentralization is one of the core concepts or features of a blockchain. It can mean different things in different
contexts but for our purposes it allows for (1) decentralization of control or power, that, (2) decentralization of
software.

{\em Why is decentralization useful ?}
\begin{itemize}

    \item Fault tolerance - decentralized systems are less likely to fail accidentally because they rely on many separate
components that are not likely.
    \item Attack resistance - decentralized systems are more expensive to attack and destroy or manipulate because they lack
sensitive central points that can be attacked at much lower cost than the economic size of the surrounding system. This
can be important for transportation data as it does not remain under a single point of failure.
    \item Collusion resistance:it is much harder for participants in decentralized systems to collude to act in ways that
benefit them at the expense of other participants, whereas the leaderships of corporations and governments collude in
ways that benefit themselves but harm less well-coordinated citizens, customers, employees and the general public all
the time.
\end{itemize}

One of the key concepts of blockchains becoming an application platform is smart contracts. A standard contract, as a
legal document, binds two or more parties into an obligation to achieve certain deliverables or outcomes. A smart
contract is a piece of code that similarly binds multiple parties into outcomes that are verifiable, computable or
provable using code and strong cryptographic constructs. Ethereum was the first such platform to introduce the concept
of smart contracts which has since been adopted by most other blockchains that wish to host decentralized applications
(dapps).

Because smart contracts can be executed by arbitrary nodes on the blockchain, its possible for anyone to "verify" the
smart contract. This createst the concept of trust using consensus. Consensus is defined as the agreement among a
certain number (or fraction) of nodes on a particular result or outcome. With consensus, it becomes much harder for a
Byzantine or adversarial node \cite{lamport_byz} to manipulate the smart contract in ways that was not intended or
provisioned for. One of our goals, as discussed later in Section \ref{sec:design} is to build a smart contract framework
that is tailored for Transportation applications, so that the contracts are readily available, trustable and
enforceable. 

Smart contracts can allow new ways of data sharing that were not possible before. By creating strong programmatic
constructs combined with consensus protocols and cryptographic primitives it is possible to share data in a way that
privacy, security and restrictions on use can be enforced. There has been a lot of recent progress on how to use Smart
Contracts for data sharing \cite{liu_2018}. Thus, the technology today is ready for creating disruptive new ways of
sharing transportation data to create new applications such as smart cities, driver behavior, insurance, mapping etc.
This technology can be disruptive to incumbents who might be late for adoption.

- Concepts of smart contract, trust, consensus.

- How data sharing can work in a blockchain manner. 
- Overview of cryptographic proofs. How they work.
- Discussion of Byzantine behavior.

-Blockchain based Platform is decentralized.
    - No single point of control or failure.
    - Users own their data and reputation.
    - Every functionality is openly verifiable.

- System is open-source
 - Every entity knows how data, algorithms and logic is implemented.
 - Policies can be verified using Trusted Computing.

- Anyone can participate including Governments, Individuals, Enterprises.
- Governance happens using a council of participants.
- Privacy and Security policies can be enforced through strong cryptographic primitives.


- Smart contracts allow for precise enforcement of incentives, data sharing and privacy protection.
- Industry or Government Standards are enforceable and verifiable.
- Raw and derived data can outlive the companies or geographies.
- Eg: DriverScore can carry over to a different country or geography.
- Mapping data can be utilized outside of the GIS provider.


